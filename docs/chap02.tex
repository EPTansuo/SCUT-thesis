\chapter{第二章	卷积神经网络的基础知识}
\section{卷积神经网络的网络结构}
卷积神经网络作为深度学习的一个分支,在网络结构上同样含有深度学习的“深度”性。网络拓扑结构是一个多层的神经网络\overcite{ref8},网络的每一层由多个独立的神经元组成的二维平面组成。网络一般分为输入层、卷积层、池化层、全连接层、输出层等。
\subsection{输入层}
因为卷积神经网络可以直接的接受二维的视觉模式\overcite{ref9},所以我们可以直接把简单预处理后的二维图像输入到输入层中。
\subsection{输出}
……
\section{卷积神经网络的学习规律}
……
\subsection{前向传播}
如果用$l$来表示当前的网络层,那么当前网络层的输出如\autoref{eq:fp}所示:
\begin{equation}
    \label{eq:fp}
    {x^l} = f({u^l}),\text{其中}{u^l} = {W^l}{x^{l - 1}} + {b^l}
\end{equation}
其中$f(\cdot)$为网络的输出激活函数。在本文实验中,网络的输出激活函数选用sigmoid函数,因此网络的输出均值一般来说趋于0。
\subsection{反向传播}
……
\subsection{学习特征图的组合}
……
\section{本章小结}
……